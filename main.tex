\documentclass[journal,a4paper,twoside]{template/IEEEtran}

\usepackage[utf8]{inputenc}
\usepackage[english]{babel}
\usepackage{template/EVjour}

% *** GRAPHICS RELATED PACKAGES ***
\ifCLASSINFOpdf
   \usepackage[pdftex]{graphicx}
\else % or other class option (dvipsone, dvipdf, if not using dvips).
   \usepackage[dvips]{graphicx}
\fi
%\usepackage{epsfig}
\usepackage{epstopdf} % A.T. include eps images in pdftex (Miktex2.8 or higher)

% Include all other packages here
\usepackage{textcomp} % need for \textmu
%\usepackage{eurosym} %  symbol \euro
%\usepackage[cmex10]{amsmath} % *** MATH PACKAGES ***

% correct bad hyphenation here
%\hyphenation{op-tical net-works semi-conduc-tor}

\begin{document}

% naslov prispevka, lahko uporabite en linebreak \\
\title{Mitigating Cross-Site Request Forgery Attacks}

\authors{Matjaž Mav}

\address{University of Ljubljana, Faculty of Computer and Information Science, Večna pot 113, 1000 Ljubljana, Slovenia\\
E-mail: mm3058@student.uni-lj.si}

\abstract{TODO}

\keywords{Cross-Site Request Forger, CSRF, XSRF, XSFR, OAuth 2.0}

% Priimki avtorjev in kratek naslov članka za tekočo glavo
\markboth{Mav}{Mitigating Cross-Site Request Forgery Attacks}

% make the title area
\maketitle

\IEEEpeerreviewmaketitle

\section{Introduction}
\label{sec_introduction}

Cross-Site Request Forgery is an attack where authenticated user executes unwanted action of attacker's choosing. This attacks usually target state-changing requests (for example executing unwanted bank transaction) and not compromising data secrecy. But it is also possible to attack the login process and this can lead to private data exposure \cite{barth2008robust}.

In this paper, we will use CSRF as an abbreviation for Cross-Site Request Forgery. In other texts, we can notice XSRF or XSFR used as synonyms for CSRF abbreviation.

The basis for this paper is an article called \textit{Mitigating CSRF attacks on OAuth 2.0 Systems} \cite{li2018mitigating} published in 2018.

According to the OWASP\footnote{Open Web Application Security Project} Top 10 report from the year 2013 \cite{owasptopten}, cross-site request forgery attacks ware ranked at number 8 of top 10 most critical web application security risks. However, in the most recent OWASP Top 10 report form year 2017, CSRF is no longer on the list. Few of the reasons for its fall from the top 10 list are that \cite{owaspthefallofcsrf} (1) more frameworks are now offering secure-by-default settings, (2) new and improved browser standards for CSRF attack mitigation and (3) overall raised awareness for secure development.

\section{CSRF background}
\label{sec_csrfbackground}

In this section we will go through simple CSRF attack example, where authenticated user unintentionally executes bank transaction (figure \ref{img_simplecsrfattack}).

First user sends (1) authentication request to the online bank (\textit{bank.com}) and receives back (2) authentication cookie. Then he opens seemingly friendly email and (3) clicks on the link. Link opens browser and (4) loads attacker site (\textit{attacker.com}). Without user knowledge, this malicious site sends HTTP request to the bank and (5) create bank transaction. Because user's browser still has valid authentication cookie, the malicious HTTP request is automatically accompanied with authentication cookie.

\begin{figure}[htb]
\centerline{\includegraphics[width=0.4\paperwidth]{resources/CSRF_example.png}}
\caption{The simple example of a CSRF attack.}
\label{img_simplecsrfattack}
\end{figure}

To mitigate CSRF attacks we commonly use some combination of the following approaches:
\begin{enumerate}
    \item The first approach uses randomly generated tokens that are sent at each HTTP request. These tokens are randomly generated on the server and then two copies are sent back to the browser. One copy is stored inside the cookie and the other one is injected into form. When the form is submitted, the server compares value stored inside a cookie with value submitted over form. If these two values are valid and match request is further processed, otherwise request is rejected \cite{li2018mitigating}.
    \item The second approach prevent execution requests that are initiated from different origins. This can be achieved using either (1) \texttt{Referer} request header, (2) custom HTTP headers or (3) \texttt{Origin} request header \cite{barth2008robust,li2018mitigating}. Each of this approaches can distinguish between same-origin or cross-origin requests.
    \item The third and most recent approach \cite{owaspthefallofcsrf} is to use authentication cookie with \texttt{SameSite} attribute \cite{Goodwin2018Nov}. Cookies annotated with \texttt{SameSite} attribute are only   .
\end{enumerate}

The second and third approach heavily relies on the specific browser and thus cannot be completely trusted.%The first approach requires HTTPS connection because the same token can be used multiple times and can be intercepted. Also, if our site is vulnerable to Cross-Site Scripting (XSS) attacks these mitigating techniques won't work.

\section{OAuth 2.0 background}

\section{Attack on OAuth 2.0}

\section{Conclusions}

\small
\bibliographystyle{plain}
\bibliography{bibliography}

\vfill

\end{document}